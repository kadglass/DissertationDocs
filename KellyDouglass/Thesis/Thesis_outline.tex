\documentclass{article}

\begin{document}

% Title
\title{Thesis outline}
\author{Kelly Douglass, Michael S. Vogeley, advisor}
\date{21 June 2016}

\maketitle


%%%%%%%%%%%%%%%%%%%%%%%%%%%%%%%%%%%%%%%%%%%%%%%%%%%%%%%%%%%%%%%%%%%%%%%%%%%%%%%%
% Outline

\section{Large-scale environmental dependence of gas-phase metallicity in dwarf galaxies}
\begin{itemize}
    \item Submission dates
    \begin{itemize}
        \item Submitted April 2016
        \item To be resubmitted June 2016
    \end{itemize}
    \item Remaining work
    \begin{itemize}
        \item Apply referee suggestions
    \end{itemize}
\end{itemize}

\section{Large-scale environmental dependence of the abundance ratio of nitrogen to oxygen in dwarf galaxies}
\begin{itemize}
    \item Submission date --- July 2016
    \item Remaining work
    \begin{itemize}
        \item Revisions to make publication-ready
    \end{itemize}
\end{itemize}

\section{Approximation for the abundance of singly-ionized oxygen in SDSS galaxies}
\begin{itemize}
    \item Submission date --- September 2016
    \item Remaining work
    \begin{itemize}
        \item The purpose of this paper will be to use the approximation to bolster the low-mass end of my galaxy sample, with which I can therefore do full comparisons with previous works.  It will concentrate on both the dwarf galaxy characteristics as well as the full galaxy trends (in N/O).  
        \item Repeat dwarf galaxy large-scale environment analysis (oxygen, nitrogen, N/O) with ~2000 dwarf galaxies with approximated abundances (abundances already calculated)
        \item Finish composing paper, revisions for publication
    \end{itemize}
\end{itemize}

\section{Large-scale environmental dependence of gas-phase abundances in dwarf galaxies from SHELS}
\begin{itemize}
    \item Submission date --- Winter/Spring 2017
    \item Remaining work
    \begin{itemize}
        \item The direct method does not work with this data, since the S/N is too low to measure the [O{\sc iii}] $\lambda 4363$ auroral line.  I need to settle on a second-best method for abundance calculations that can be used on these galaxies (probably those from Brown, Martini, and Andrews (2016)).
    \end{itemize}
\end{itemize}

\section{(possible) Small-scale environmental dependence of gas-phase abundances in dwarf galaxies}
\begin{itemize}
    \item Submission date --- Winter/Spring 2017
    \item Remaining work
    \begin{itemize}
        \item Look through Brean's code and work and pick up where she left off.
    \end{itemize}
\end{itemize}

\section{(possible) The chemical evolution of AGNs}
\begin{itemize}
    \item Submission date --- Spring 2017
    \item Remaining work
    \begin{itemize}
        \item The oxygen abundance estimates for AGN are normal, but the nitrogen abundance is much higher than in star-forming galaxies.  I want to see if I can determine the cause of this discrepancy, so that the same analysis on the abundances can be done on AGN, including any large-scale dependence.
    \end{itemize}
\end{itemize}

\end{document}