\chapter[Direct method physics]{The physics of the Direct $T_e$ method}

UV photons from young stars in an \ion{H}{2} region keep the interstellar gas 
partially ionized.  In an \ion{H}{2} region, a photon with an energy 
$E_\gamma \gtrsim 13.6\text{ eV}$ ionizes a hydrogen atom, producing an H$^+$ 
ion and an electron with some kinetic energy 
$K_{e^-} = E_\gamma - 13.6\text{ eV}$.  As the free electron moves around in the 
gas, it loses some of its kinetic energy as it collisionally excites other ions 
in the gas.  Because we are in a Str\"{o}mgren sphere (an ionized region or an 
\ion{H}{2} region), the cross-section for this electron scattering is much 
larger than the photoionization cross section, so the free electrons will 
thermalize quickly \citep{DeRobertis87}.  Eventually, the free electron 
recombines with an H$^+$ ion, forming atomic hydrogen again.  The ions that were 
collisionally excited by the electron before its recombination eventually 
de-excite via forbidden transitions, emitting radiation that escapes the 
\ion{H}{2} region.

In equilibrium, the total energy input into the gas via radiation from the star 
(that ionizes the hydrogen), $E_{photoionization}$, must be equal to the energy 
required to recombine the electron and the H$^+$ ion, $E_{recombination}$, and 
the energy radiated away from the \ion{H}{2} region by the forbidden 
transitions, $E_{radiation}$.  So
\begin{equation}
    E_{photoionization} = E_{recombination} + E_{radiation}
\end{equation}
For radiative de-excitation to dominate over collisional de-excitation, the 
electron density $n_e \ll 10^{8-10}\text{ cm}^{-3}$ \citep{DeRobertis87}.  
Typical \ion{H}{2} regions have an electron density 
$n_e \approx 100\text{ cm}^{-3}$, so radiative de-excitation dominates; this is 
why we observe forbidden transitions in outer space but not here on Earth.


\section[Estimating the temperature]{Temperatures from the forbidden emission lines}

%%%%%%%% All from DeRobertis87
%Of the ions that dominate radiative cooling (O$^+$, N$^+$, etc.), most have 
%either p$^2$, p$^3$, or p$^4$ ground-state electron configurations with five 
%energy levels $\leq 5\text{ eV}$ above the ground state.  For each excitation 
%level $i$, \cite{DeRobertis87} shows that the equations of statistical 
%equilibrium are
%\begin{equation}
%    \sum_{j \neq i} n_j n_e q_{ij} + \sum_{j > i} n_j A_{ji} = \sum_{j \neq i} n_i n_e q_{ij} + \sum_{j < i} n_i A_{ij}
%\end{equation}
%where $n_j$ is the fraction of the ion in level $j$, $q_{ij}$ is the electron 
%(de)excitation rate coefficient (in units of cm$^{-3}$s$^{-1}$), and $A_{ij}$ is 
%the radiate transition probability with units of s$^{-1}$.  From left to right, 
%these terms describe the collisional transitions to $i$, the radiative 
%transitions down to $i$, the collisional transitions from $i$, and the radiative 
%transitions down from $i$.  For a given ion X$^i$ with a number density 
%$N(\text{X}^i)$, the relative populations must must sum to unity:
%\begin{equation}
%    \sum_j \frac{n_j}{N(\text{X}^i} = 1
%\end{equation}
%
%$A_{ij}$ are independent of temperature and inversely proportional to the 
%lifetime of the upper level.  Conversely, the coefficients $q_{ij}$ depend on 
%temperature since they are a measure of how likely collisions will occur.  These 
%can be calculated as
%\begin{equation}
%    q_{ij} = \frac{8.629\times 10^{-6}}{T_e^{1/2}} \frac{\Omega (i,j)}{\omega_i}
%\end{equation}
%where $T_e$ is the electron temperature, $\omega_i$ is the statistical weight of 
%level $i$, and $\Omega (i,j)$ is the average collisional strength.  While 
%$\Omega$ is dimensionless, it does depend on the temperature 
%\cite{DeRobertis87}.  The excitation rate $q_{ij}$ is related to the 
%de-excitation rate $q_{ji$ by
%\begin{equation}
%    q_{ji} = \frac{\omega_i}{\omega_j} q_{ij} \exp{-\Chi_{ij}/kT}
%\end{equation}
%where $\Chi_{ij}$ is the excitation energy difference between levels $i$ and 
%$j$.


%%%%%%% From Osterbrock89

%In a pure hydrogen nebula, \cite{Osterbrock89} shows us that the photoionization 
%energy input (per unit volume per unit time) is equal to 
%\begin{equation}
%    G(H) = n_e n_p \alpha_A (H^0, T) \frac{3}{2} kT_i
%\end{equation}
%where $n_e$ is the number density of electrons, $n_p$ is the number density of 
%protons, $\alpha_A$ is the recombination coefficient summed over all energy 
%levels, and $T_i$ is the photoelectron's initial temperature.  Assuming a 
%blackbody spectrum, $T_i \approx T_*$ for $kT_* < h\nu_0$.  The gain $G(H)$ 
%depends on the form of the ionizing radiation field, not its strength.  The rate 
%of creation of the photoelectrons depends on the strength of the radiation 
%field, or the recombination rate.
%
%Likewise, \cite{Osterbrock89} shows that the rate of energy lost through 
%recombination per unit volume is equal to
%\begin{equation}
%    L_R (H) = n_e n_p kT \beta_A (H^0, T)
%\end{equation}
%where $\beta_A$ is the effective recombination coefficient for recombination 
%energy loss.

Expanding the description in Sec. \ref{sec:O3}, we can define the ratio of the 
emission line strengths for [\ion{O}{3}] $\lambda$4363 and [\ion{O}{3}] 
$\lambda \lambda$4959,5007 as
\begin{equation}
    \frac{j_{4959} + j_{5007}}{j_{4363}} = \frac{\Omega(^3\text{P}, ^1\text{D})}{\Omega(^3\text{P}, ^1\text{S})} \left[ \frac{A_{^1\text{S}, ^1\text{D}} + A_{^1\text{S}, ^3\text{P}}}{A_{^1\text{S}, ^1\text{D}}} \right] \frac{\bar{\nu}(^3\text{P}, ^1\text{D})}{\nu(^1\text{D}, ^1\text{S})} e^{\Delta E/kT}
\end{equation}
where $j_\lambda$ is the emissivity of the emission line, $\Omega (i,j)$ is the 
collision strength between levels $i$ and $j$, $A_{i,j}$ is the radiative 
transition probability between an upper level $i$ and a lower level $j$, $\nu$ 
is the frequency of the transition, $\Delta E$ is the energy difference between 
the $^1$D$_2$ and $^1$S$_0$ levels, and $T$ is the electron temperature 
\cite{Osterbrock89}.  The transition probabilities $A_{i,j}$ do not depend on 
the temperature, but the collision strength $\Omega (i,j)$ is 
temperature-dependent.  We can define
\begin{equation}
    \bar{\nu}(^3\text{P}, ^1\text{D}) = \frac{A_{^1\text{D}_2, ^3\text{P}_2} \nu (5007) + A_{^1\text{D}_2, ^3\text{P}_1} \nu (4959)}{A_{^1\text{D}_2, ^3\text{P}_2} + A_{^1\text{D}_2, ^3\text{P}_1}}
\end{equation}
Inserting numerical values for the collision strengths and transition 
probabilities from \cite{Osterbrock89}, the ratio becomes
\begin{equation}
    \frac{j_{4959} + j_{5007}}{j_{4363}} = \frac{7.73 \exp[(3.29\times 10^4)/T]}{1 + 4.5\times 10^{-4} (N_e/T^{1/2})}
\end{equation}
By measuring the flux of these emission lines, we can then solve for the 
temperature of the gas.


%\section{[\ion{O}{3}]}
%
%There are three significant emission lines for doubly-ionized oxygen.  The 
%relative excitation rates to the $^1$S and $^1$D energy levels depend very 
%strongly on the electron temperature, $T_e$; therefore, the relative strengths 
%of these emitted lines can be used to measure the electron temperature 
%\citep{Osterbrock89}.  In the low-density limit $(n_e < 10^5 \text{ cm}^{-3})$, 
%most excitations to the $^1$D level result in an emission of a photon with a 
%wavelength of either $5007\text{\AA}$ or $4959\text{\AA}$, as shown in Fig. 
%\ref{fig:transitions_P1}.  Most excitations up to $^1$S produce a photon of 
%wavelength $4363\text{\AA}$, followed by a photon of either of the two previous 
%wavelengths (since the electron is now in the $^1$D level).
%
%At higher densities, collisional de-excitation begins to influence these 
%emission rates \citep{Osterbrock89}.  Because the $^1$D level has a longer 
%lifetime than the $^1$S state, it is collisionally de-excited at lower electron 
%densities.  This weakens the $\lambda 4959$ and $\lambda 5007$ emission lines.  
%At the same time, the additional collisional excitations of the $^1$D state 
%permitted by the higher electron densities strengthen the $\lambda 4363$ 
%emission line.
%
%[\ion{O}{3}] $\lambda 4363$ is a temperature-sensitive forbidden transition line 
%of doubly-ionized oxygen that is the preferred line to use when measuring the 
%metallicity of galaxies.  Since the most effective cooling channel in \ion{H}{2} 
%regions is oxygen line emission, lower metallicity regions have higher 
%temperatures \citep{Saintonge07}.  Collisional excitations up to this energy 
%level are more common at higher temperatures, since there are more electrons 
%with the kinetic energy required to excite the O$^{++}$ ion to this energy 
%level.  As a result, the line strength of [\ion{O}{3}] $\lambda 4363$ correlates 
%with the region's temperature and is therefore anticorrelated with the 
%metallicity of the galaxy.


%\section{[\ion{N}{2}]}
%
%The energy-level diagram for the various transitions of [\ion{N}{2}] is very 
%similar to that of [\ion{O}{3}], since they have the same electron ground-state 
%configuration ((1s)$^2$(2s)$^2$(2p)$^2$).  The similarities can be seen in Fig. 
%\ref{fig:transitions_P2}.  Therefore, an estimate of the electron temperature can 
%be made from the [\ion{N}{2}] $\lambda 5755$ emission line.  However, this line 
%is weaker than the [\ion{O}{3}] $\lambda 4363$ auroral line (since there is less 
%N than O in galaxies), so we use the [\ion{O}{3}] auroral line for our 
%temperature estimates, as in Paper I.  After obtaining a temperature and density 
%estimate, we use the [\ion{N}{2}] $\lambda \lambda 6548, 6584$ doublet to 
%estimate the abundance of singly ionized nitrogen in a galaxy.


\section{[\ion{S}{2}]}

Similar to the sensitivity of the doubly-ionized oxygen ion transitions to the 
electron temperature of the surrounding gas, the transitions for singly-ionized 
sulfur are sensitive to electron number density.  The relative excitation rates 
depend only on the ratio of the collision strengths when two emission lines 
(from the same ion) with nearly identical excitation energies are compared 
\citep{Osterbrock89}.  If the two levels have different transition probabilities 
and/or different collisional de-excitation rates, their ratio depends on the 
density.

The relative excitation rates of the two lines shown in Fig. 
\ref{fig:transitions_P1} are proportional to their statistical weights; thus, the 
ratio of the line intensities is a constant in the low-density limit 
\citep{Osterbrock89}.  In the high-density regime, this ratio is best accurately 
described by a Boltzmann population ratio.  There is a critical density for the 
energy levels which describes the turning point between these two extremes.