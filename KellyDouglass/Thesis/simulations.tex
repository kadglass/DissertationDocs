\chapter[Simulations]{Simulations of Dwarf Galaxy Formation and Evolution}


Numerous simulations have been performed over the years in an attempt to 
understand the formation and evolution of our universe.  As observers, it is 
important to connect our observations with the theory in order to grow our ideas 
about the universe.


\section{Dwarf galaxy formation and evolution}

% Hierarchical formation of halos
It is impossible to directly observe the formation of a galaxy.  Based on 
detected merging galaxies, the astronomical community has surmised that galaxies 
have increased their mass largely due to the merging of smaller galaxies.  
$\Lambda$CDM simulations of hierarchical galaxy formation by \cite{deRossi07} 
find that the growth of systems by aggregation is responsible for the 
mass-metallicity relation.  Less massive systems tend to increase their stellar 
mass by either gas-rich mergers or secular evolution, resulting in a stronger 
correlation between stellar mass and metallicity.  Evidence of this is seen in 
the mass-metallicity relation of \cite{Tremonti04}, where there is little 
relationship between stellar mass and the metallicity for massive systems, but 
at intermediate and lower masses the correlation is strong between the two 
quantities.

% ISM - inflows and outflows
The inflow and outflow of gas in galaxies are aspects of galaxy evolution that 
are not yet well understood.  This is especially important when studying dwarf 
galaxies, since their gravitational potential wells may be shallow enough to be 
strongly influenced by stellar winds and/or supernovae.  Simulations by 
\cite{Marcolini04} find that when ram pressure from the intergalactic medium 
(IGM) is comparable to the thermal pressure of the central interstellar medium 
(ISM), stripping and superwind influence each other and increase the gas removal 
rate.  They find that the amount of metal-rich ejecta is sensitive to the ram 
pressure.  
% supernova
Simulations by \cite{Power14} and \cite{Melioli15} show that supernovae contain 
sufficient energy to unbind the gas in a low mass halo.  \cite{Hu16} show that 
supernova-driven galactic outflows push metal-rich gas into the halo.  A small 
fraction of the gas eventually escapes the halo, while most falls back onto the 
disc.  This results in the halo metallicity being about 20\% higher than the 
disc metallicity.  Likewise, \cite{Muratov17} find that almost all metals 
produced in a Type II supernova are ejected from the galaxy.
% star formation
Unlike the supernovae, \cite{Melioli15} finds that the ISM material does not 
have a strong likelihood of being expelled from a galaxy due to star formation.  
In contrast, \cite{Muratov15} finds that a large portion of material is ejected 
in galactic outflows after a bust of star formation, which collects in the 
circumgalactic medium (CGM) and is sometimes recycled in later star formation 
episodes.  In galaxies with $M_h < 10^{12} M_\odot$, a fraction of material is 
lost to the IGM.

% Metallicity
Numerical simulations that study the properties of the ISM in an isolated, 
star-forming galaxy find that stellar feedback slowly increases the disc 
metallicity \citep{Hu16}.  The majority of outflows are enriched winds, which 
reduces the metallicity of the ISM.  In dwarf galaxies at low redshifts, these 
outflows to the CGM are dominated by metal-poor gas \citep{Muratov17}.  While 
galaxies retain most of the metals they produce, a large fraction is in the CGM.  
By combining our observations with the results of these simulations, we can 
begin to understand how star formation and supernovae influence a dwarf galaxy's 
gas content.


\section[Environmental differences]{Environmental differences of galaxy formation and evolution}

Unlike the simulations described above, simulations which investigate the 
influence of the large-scale environment on the formation and evolution of 
galaxies have presented a more uniform picture.  Simulations by \cite{Jung14,
Xie14} and \cite{Tonnesen15} show that dark matter halos assembled later in 
underdense regions for a given halo mass.  For central galaxies, \cite{Jung14} 
and \cite{Tonnesen15} find that the ratio of stellar mass to halo mass is larger 
in overdense regions because star formation rates were higher in these denser 
regions.

\cite{Cen11} studied cosmic downsizing with high-resolution large-scale 
hydrodynamic galaxy formation simulations based on the $\Lambda$CDM cosmology.  
He finds that cosmic downsizing is part of a trend where the sSFR is a function 
of halo mass such that lower-mass galaxies have higher sSFRs.  The formation of 
large halos (groups and clusters) and the large-scale structure causes cosmic 
gas to heat beyond a critical temperature at which the gas's entropy is too high 
to cool to continue feeding the galaxies and their star formation.  As the 
surrounding gas in the dense regions becomes heated beyond the critical limit, 
the sSFR of the affected galaxies decreases and the galaxies transition from the 
blue sequence to the red cloud.  In the void environment, this heating does not 
occur, and so galaxies in the voids maintain a high sSFR in the present epoch.  
While this is the overall trend, there still exists some galaxies at $z=0$ which 
have a high sSFR but reside in the more dense environments.