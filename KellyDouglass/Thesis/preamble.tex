\begin{preamble}

\iffinal{}{\newpage}

%%%%%%%%%%%%%%%%%%%%%%%%%%%%%%%%%%%%%%%%%%%%%%%%%%%%%%%%%%%%%%%%%%%%%%%%%%%%%%%%
% Dedications
%%%%%%%%%%%%%%%%%%%%%%%%%%%%%%%%%%%%%%%%%%%%%%%%%%%%%%%%%%%%%%%%%%%%%%%%%%%%%%%%

\begin{DUTdedications}
\begin{center}
To my parents\\
You always encouraged me to reach for the stars\\
Will galaxies do?
\end{center}
\end{DUTdedications}

\iffinal{}{\newpage}

%%%%%%%%%%%%%%%%%%%%%%%%%%%%%%%%%%%%%%%%%%%%%%%%%%%%%%%%%%%%%%%%%%%%%%%%%%%%%%%%
% Acknowledgements
%%%%%%%%%%%%%%%%%%%%%%%%%%%%%%%%%%%%%%%%%%%%%%%%%%%%%%%%%%%%%%%%%%%%%%%%%%%%%%%%

\begin{acknowledgments}

% Advisor
I would like to begin by thanking my advisor, Prof. Michael Vogeley, for being a 
fantastic advisor and role model.  Thank you for helping me expand my knowledge 
and research skills over the last six years.  And thank you for giving me the 
opportunity to help advise students.  I would also like to thank our 
collaborator, Renyue Cen, for his help in the theoretical interpretation of 
these results and for confirming that our explanations are not completely 
ridiculous.  To my committee members: Profs. Gordon Richards, Steve McMillan, 
Masao Sako, and Michelle Dolinski --- thank you for your time and guidance 
throughout the years.  And a big shout-out to my advisees: Daniele Schneider, 
Jinfu Dai, Brean Prefontaine, and Salvatore Zerbo.  Thank you for all your hard 
work!

% Grad students & PLBD
Thank you to all the graduate students in the Drexel Physics department for the 
wonderful, supportive, collaborative community.  You helped make grad school so 
much more bearable (and almost enjoyable, if I dare say).  To my physics boys 
Justin, Matt, and Matt: we made it!  I would also be remiss in not acknowledging 
my office mates over the last few years, particularly Josh, Jackie, Justin, 
Crystal, Austen, and Vish, for help with code, ideas, and keeping up the moral.  
And, of course, thank you to all the members of PLBD for providing me an outlet 
from physics.  You guys are the best!

% Family
None of this would have been possible without the infinite love and support from 
my family.  Thank you Dad, Mom, Kelly, and Ren\'{e}e for always pushing me to 
work my hardest and to do my best.  Thank you for challenging me in so many 
ways.  I love you guys, and it means so much that you are here today!

% Steve
To Steve, my dance partner --- thank you for understanding the time and energy 
a thesis takes to write.  And to Steve, my not dance partner --- thank you for 
helping and supporting me in any and all ways possible throughout this process; 
you made this so much easier.
\end{acknowledgments}

\iffinal{}{\newpage}

%%%%%%%%%%%%%%%%%%%%%%%%%%%%%%%%%%%%%%%%%%%%%%%%%%%%%%%%%%%%%%%%%%%%%%%%%%%%%%%%
% Table of Contents
%%%%%%%%%%%%%%%%%%%%%%%%%%%%%%%%%%%%%%%%%%%%%%%%%%%%%%%%%%%%%%%%%%%%%%%%%%%%%%%%

\tableofcontents

\iffinal{}{\newpage}

%%%%%%%%%%%%%%%%%%%%%%%%%%%%%%%%%%%%%%%%%%%%%%%%%%%%%%%%%%%%%%%%%%%%%%%%%%%%%%%%
% List of Tables
%%%%%%%%%%%%%%%%%%%%%%%%%%%%%%%%%%%%%%%%%%%%%%%%%%%%%%%%%%%%%%%%%%%%%%%%%%%%%%%%

\listoftables

\iffinal{}{\newpage}

%%%%%%%%%%%%%%%%%%%%%%%%%%%%%%%%%%%%%%%%%%%%%%%%%%%%%%%%%%%%%%%%%%%%%%%%%%%%%%%%
% List of Figures
%%%%%%%%%%%%%%%%%%%%%%%%%%%%%%%%%%%%%%%%%%%%%%%%%%%%%%%%%%%%%%%%%%%%%%%%%%%%%%%%

\listoffigures

\iffinal{}{\newpage}

%%%%%%%%%%%%%%%%%%%%%%%%%%%%%%%%%%%%%%%%%%%%%%%%%%%%%%%%%%%%%%%%%%%%%%%%%%%%%%%%
% Abstract
%%%%%%%%%%%%%%%%%%%%%%%%%%%%%%%%%%%%%%%%%%%%%%%%%%%%%%%%%%%%%%%%%%%%%%%%%%%%%%%%

\begin{abstract}

% Papers 1-3
We investigate how the cosmic environment affects galaxy evolution in the 
Universe by studying gas-phase chemical abundances and other galaxy properties 
as a function of the large-scale environment and local density of galaxies.  
Using spectroscopic observations from the Sloan Digital Sky Survey Data Release 
7, we estimate the oxygen and nitrogen abundances of 993 star-forming void dwarf 
galaxies and 759 star-forming dwarf galaxies in denser regions.  We use the 
Direct $T_e$ method for calculating the gas-phase chemical abundances in the 
dwarf galaxies because it is best suited for low metallicity, low mass galaxies.  
A substitute for the [\ion{O}{2}] $\lambda$3727 doublet is developed, permitting 
oxygen abundance estimates of SDSS dwarf galaxies at all redshifts with the 
Direct $T_e$ method.  We find that star-forming void dwarf galaxies have 
slightly higher oxygen abundances than star-forming dwarf galaxies in denser 
environments, but we find that void dwarf galaxies have slightly lower nitrogen 
abundances and lower N/O ratios than galaxies in denser regions.  
% Small-scale enivronment
At smaller scales, we find that only the presence of a neighboring galaxy within 
0.05 \hMpc or 0.1$r_{vir}$, or the presence of a group within 0.05 \hMpc, 
influences a dwarf galaxy's evolution.  Dwarf galaxies within 0.05 \hMpc or 
0.1$r_{vir}$ of another galaxy tend to be bluer, have higher sSFRs, have higher 
oxygen abundances, and have lower N/O ratios than average.  In contrast, 
galaxies within 0.05 \hMpc of the center of the closest group have lower oxygen 
and nitrogen abundances than average.
% GV
We also investigate how a galaxy transitions through the color-magnitude 
diagram, evolving from a blue, star-forming spiral or irregular galaxy in the 
blue sequence to a red elliptical galaxy in the red cloud through the green 
valley.  We discover that combining a galaxy's color, color gradient, and 
inverse concentration index determines a galaxy's location on the 
color-magnitude diagram.  The results indicate that, in the green valley, there 
is a lower fraction of void dwarf galaxies than dwarf galaxies in denser 
regions.
% Conclusions
From these analyses, we surmise that void dwarf galaxies experience delayed star 
formation as predicted by the $\Lambda$CDM cosmology.  We also conjecture that 
cosmic downsizing corresponds to a shift towards star formation in both lower 
mass objects and void regions closer to the present epoch.  We present evidence 
that void dwarf galaxies may have a higher ratio of dark matter halo mass to 
stellar mass when compared to dwarf galaxies in denser environments.

\end{abstract}

\iffinal{}{\newpage}

\end{preamble}
