\begin{preamble}

\iffinal{}{\newpage}

%%%%%%%%%%%%%%%%%%%%%%%%%%%%%%%%%%%%%%%%%%%%%%%%%%%%%%%%%%%%%%%%%%%%%%%%%%%%%%%%
% Dedications
%%%%%%%%%%%%%%%%%%%%%%%%%%%%%%%%%%%%%%%%%%%%%%%%%%%%%%%%%%%%%%%%%%%%%%%%%%%%%%%%

\begin{DUTdedications}
\begin{center}
Dedications
\end{center}
\end{DUTdedications}

\iffinal{}{\newpage}

%%%%%%%%%%%%%%%%%%%%%%%%%%%%%%%%%%%%%%%%%%%%%%%%%%%%%%%%%%%%%%%%%%%%%%%%%%%%%%%%
% Acknowledgements
%%%%%%%%%%%%%%%%%%%%%%%%%%%%%%%%%%%%%%%%%%%%%%%%%%%%%%%%%%%%%%%%%%%%%%%%%%%%%%%%

\begin{acknowledgments}

% Advisor

% Thesis committee and collaborators (Renyue, Daniele, Jinfu, Brean, Salvatore)
I would like to thank our collaborator Renyue Cen for his help in the 
interpretation of these results.  I would also like to acknowledge Crystal 
Moorman for her help and support throughout this work.  Finally, I would like to 
thank the referees (including Burcu Beygu) for their detailed comments and 
critique of my work.

% Grad students & PLBD

% Family

% Steve


%Support for this work was provided by NSF grant AST--1410525.
%
%Funding for the SDSS and SDSS-II has been provided by the Alfred P. Sloan 
%Foundation, the Participating Institutions, the National Science Foundation, the 
%U.S. Department of Energy, the National Aeronautics and Space Administration, 
%the Japanese Monbukagakusho, the Max Planck Society, and the Higher Education 
%Funding Council for England.  The SDSS Web Site is \emph{http://www.sdss.org/}.
%
%The SDSS is managed by the Astrophysical Research Consortium for the 
%Participating Institutions.  The Participating Institutions are the American 
%Museum of Natural History, Astrophysical Institute Potsdam, University of Basil, 
%University of Cambridge, Case Western Reserve University, University of Chicago, 
%Drexel University, Fermilab, the Institute for Advanced Study, the Japan 
%Participation Group, Johns Hopkins University, the Joint Institute for Nuclear 
%Astrophysics, the Kavli Institute for Particle Astrophysics and Cosmology, the 
%Korean Scientist Group, the Chinese Academy of Sciences (LAMOST), Los Alamos 
%National Laboratory, the Max-Planck-Institute for Astronomy (MPIA), the 
%Max-Planck-Institute for Astrophysics (MPA), New Mexico State University, Ohio 
%State University, University of Pittsburgh, University of Portsmouth, Princeton 
%University, the United States Naval Observatory, and the University of 
%Washington.

\end{acknowledgments}

\iffinal{}{\newpage}

%%%%%%%%%%%%%%%%%%%%%%%%%%%%%%%%%%%%%%%%%%%%%%%%%%%%%%%%%%%%%%%%%%%%%%%%%%%%%%%%
% Table of Contents
%%%%%%%%%%%%%%%%%%%%%%%%%%%%%%%%%%%%%%%%%%%%%%%%%%%%%%%%%%%%%%%%%%%%%%%%%%%%%%%%

\tableofcontents

\iffinal{}{\newpage}

%%%%%%%%%%%%%%%%%%%%%%%%%%%%%%%%%%%%%%%%%%%%%%%%%%%%%%%%%%%%%%%%%%%%%%%%%%%%%%%%
% List of Tables
%%%%%%%%%%%%%%%%%%%%%%%%%%%%%%%%%%%%%%%%%%%%%%%%%%%%%%%%%%%%%%%%%%%%%%%%%%%%%%%%

\listoftables

\iffinal{}{\newpage}

%%%%%%%%%%%%%%%%%%%%%%%%%%%%%%%%%%%%%%%%%%%%%%%%%%%%%%%%%%%%%%%%%%%%%%%%%%%%%%%%
% List of Figures
%%%%%%%%%%%%%%%%%%%%%%%%%%%%%%%%%%%%%%%%%%%%%%%%%%%%%%%%%%%%%%%%%%%%%%%%%%%%%%%%

\listoffigures

\iffinal{}{\newpage}

%%%%%%%%%%%%%%%%%%%%%%%%%%%%%%%%%%%%%%%%%%%%%%%%%%%%%%%%%%%%%%%%%%%%%%%%%%%%%%%%
% Abstract
%%%%%%%%%%%%%%%%%%%%%%%%%%%%%%%%%%%%%%%%%%%%%%%%%%%%%%%%%%%%%%%%%%%%%%%%%%%%%%%%

\begin{abstract}

% Papers 1-3
We investigate how the cosmic environment affects galaxy evolution in the 
Universe by studying gas-phase chemical abundances and other galaxy properties 
as a function of the local and cosmic density of galaxies.  Using spectroscopic 
observations from the Sloan Digital Sky Survey Data Release 7, we estimate the 
oxygen and nitrogen abundances of 993 star-forming void dwarf galaxies and 759 
star-forming dwarf galaxies in denser regions.  We use the Direct $T_e$ method 
for calculating the gas-phase chemical abundances in the dwarf galaxies because 
it is best suited for low metallicity, low mass galaxies.  A substitute for the 
[\ion{O}{2}] $\lambda$3727 doublet is developed, permitting oxygen abundance 
estimates of SDSS dwarf galaxies at all redshifts with the Direct $T_e$ method.  
We find that star-forming void dwarf galaxies have slightly higher oxygen 
abundances than star-forming dwarf galaxies in denser environments, but we find 
that void dwarf galaxies have slightly lower nitrogen abundances and lower N/O 
ratios than galaxies in denser regions.  
% Small-scale enivronment
At smaller scales, we find that only the presence of a neighboring galaxy within 
0.05 \hMpc or 0.05$r_{vir}$, or the presence of a group within 0.1 \hMpc, 
influences a dwarf galaxy's evolution.  Dwarf galaxies within 0.05 \hMpc or 
0.05$r_{vir}$ of another galaxy tend to be bluer, have higher sSFRs, and have 
higher oxygen and nitrogen abundances than average.  In contrast, galaxies 
within 0.1 \hMpc of the center of the closest group are redder, have lower 
oxygen abundances, and have higher N/O ratios than average.
% GV
We also investigate how a galaxy transitions through the color-magnitude 
diagram, evolving from a blue, star-forming spiral or irregular galaxy in the 
blue sequence to a red elliptical galaxy in the red cloud through the green 
valley.  We discover that combining a galaxy's color, color gradient, and 
inverse concentration index defines a galaxy's location on the color-magnitude 
diagram.  The results indicate that there is a higher fraction of dwarf galaxies 
in denser regions than void dwarf galaxies in the green valley.
% Conclusions
From these analyses, we surmise that void dwarf galaxies experience delayed star 
formation as predicted by the $\Lambda$CDM cosmology and a dependence of the 
large-scale environment on cosmic downsizing.  We present evidence that void 
dwarf galaxies have a higher ratio of dark matter halo mass to stellar mass when 
compared to dwarf galaxies in denser environments.

\end{abstract}

\iffinal{}{\newpage}

\end{preamble}
