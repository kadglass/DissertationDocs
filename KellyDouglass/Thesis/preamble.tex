\begin{preamble}

\iffinal{}{\newpage}

%%%%%%%%%%%%%%%%%%%%%%%%%%%%%%%%%%%%%%%%%%%%%%%%%%%%%%%%%%%%%%%%%%%%%%%%%%%%%%%%
% Dedications
%%%%%%%%%%%%%%%%%%%%%%%%%%%%%%%%%%%%%%%%%%%%%%%%%%%%%%%%%%%%%%%%%%%%%%%%%%%%%%%%

\begin{DUTdedications}
\begin{center}
Dedications
\end{center}
\end{DUTdedications}

\iffinal{}{\newpage}

%%%%%%%%%%%%%%%%%%%%%%%%%%%%%%%%%%%%%%%%%%%%%%%%%%%%%%%%%%%%%%%%%%%%%%%%%%%%%%%%
% Acknowledgements
%%%%%%%%%%%%%%%%%%%%%%%%%%%%%%%%%%%%%%%%%%%%%%%%%%%%%%%%%%%%%%%%%%%%%%%%%%%%%%%%

\begin{acknowledgments}

I would like to thank Crystal Moorman for her help and support throughout this 
work.  I would also like to acknowledge Renyue Cen for his help in the 
interpretation of these results.  Finally, I would like to thank the referees 
(including Burcu Beygu) for their detailed comments and critique of my work.

Support for this work was provided by NSF grant AST--1410525.

Funding for the SDSS and SDSS-II has been provided by the Alfred P. Sloan 
Foundation, the Participating Institutions, the National Science Foundation, the 
U.S. Department of Energy, the National Aeronautics and Space Administration, 
the Japanese Monbukagakusho, the Max Planck Society, and the Higher Education 
Funding Council for England.  The SDSS Web Site is \emph{http://www.sdss.org/}.

The SDSS is managed by the Astrophysical Research Consortium for the 
Participating Institutions.  The Participating Institutions are the American 
Museum of Natural History, Astrophysical Institute Potsdam, University of Basil, 
University of Cambridge, Case Western Reserve University, University of Chicago, 
Drexel University, Fermilab, the Institute for Advanced Study, the Japan 
Participation Group, Johns Hopkins University, the Joint Institute for Nuclear 
Astrophysics, the Kavli Institute for Particle Astrophysics and Cosmology, the 
Korean Scientist Group, the Chinese Academy of Sciences (LAMOST), Los Alamos 
National Laboratory, the Max-Planck-Institute for Astronomy (MPIA), the 
Max-Planck-Institute for Astrophysics (MPA), New Mexico State University, Ohio 
State University, University of Pittsburgh, University of Portsmouth, Princeton 
University, the United States Naval Observatory, and the University of 
Washington.

\end{acknowledgments}

\iffinal{}{\newpage}

%%%%%%%%%%%%%%%%%%%%%%%%%%%%%%%%%%%%%%%%%%%%%%%%%%%%%%%%%%%%%%%%%%%%%%%%%%%%%%%%
% Table of Contents
%%%%%%%%%%%%%%%%%%%%%%%%%%%%%%%%%%%%%%%%%%%%%%%%%%%%%%%%%%%%%%%%%%%%%%%%%%%%%%%%

\tableofcontents

\iffinal{}{\newpage}

%%%%%%%%%%%%%%%%%%%%%%%%%%%%%%%%%%%%%%%%%%%%%%%%%%%%%%%%%%%%%%%%%%%%%%%%%%%%%%%%
% List of Tables
%%%%%%%%%%%%%%%%%%%%%%%%%%%%%%%%%%%%%%%%%%%%%%%%%%%%%%%%%%%%%%%%%%%%%%%%%%%%%%%%

\listoftables

\iffinal{}{\newpage}

%%%%%%%%%%%%%%%%%%%%%%%%%%%%%%%%%%%%%%%%%%%%%%%%%%%%%%%%%%%%%%%%%%%%%%%%%%%%%%%%
% List of Figures
%%%%%%%%%%%%%%%%%%%%%%%%%%%%%%%%%%%%%%%%%%%%%%%%%%%%%%%%%%%%%%%%%%%%%%%%%%%%%%%%

\listoffigures

\iffinal{}{\newpage}

%%%%%%%%%%%%%%%%%%%%%%%%%%%%%%%%%%%%%%%%%%%%%%%%%%%%%%%%%%%%%%%%%%%%%%%%%%%%%%%%
% Abstract
%%%%%%%%%%%%%%%%%%%%%%%%%%%%%%%%%%%%%%%%%%%%%%%%%%%%%%%%%%%%%%%%%%%%%%%%%%%%%%%%

\begin{abstract}

% Paper 1
We study how the cosmic environment affects galaxy evolution in the Universe by 
comparing the metallicities of dwarf galaxies in voids with dwarf galaxies in 
more dense regions.  Ratios of the fluxes of emission lines, particularly those 
of the forbidden [\ion{O}{3}] and [\ion{S}{2}] transitions, provide estimates of 
a region's electron temperature and number density.  From these two quantities 
and the emission line fluxes [\ion{O}{2}] $\lambda 3727$, [\ion{O}{3}] 
$\lambda 4363$, and [\ion{O}{3}] $\lambda \lambda 4959,5007$, we estimate the 
abundance of oxygen with the Direct $T_e$ method.  We estimate the metallicity 
of 42 blue, star-forming void dwarf galaxies and 89 blue, star-forming dwarf 
galaxies in more dense regions using spectroscopic observations from the Sloan 
Digital Sky Survey Data Release 7, as re-processed in the MPA-JHU value-added 
catalog.  We find very little difference between the two sets of galaxies, 
indicating little influence from the large-scale environment on their chemical 
evolution.  Of particular interest are a number of extremely metal-poor dwarf 
galaxies that are less prevalent in voids than in the denser regions.

% Paper 2
We examine how the cosmic environment affects the chemical evolution of galaxies 
in the universe by comparing the N/O ratio of dwarf galaxies in voids with 
dwarf galaxies in denser regions.  Ratios of the forbidden [\ion{O}{3}] and 
[\ion{S}{2}] transitions provide estimates of a region's electron temperature 
and number density.  We estimate the abundances of oxygen and nitrogen using 
these temperature and density estimates and the emission-line fluxes 
[\ion{O}{2}] $\lambda 3727$, [\ion{O}{3}] $\lambda \lambda 4959, 5007$, and 
[\ion{N}{2}] $\lambda \lambda 6548, 6584$ with the direct $T_e$ method.  Using 
spectroscopic observations from the Sloan Digital Sky Survey Data Release 7, we 
are able to estimate the N/O ratio in 42 void dwarf galaxies and 89 dwarf 
galaxies in denser regions.  The N/O ratio for void dwarfs ($M_r > -17$) is 
slightly lower ($\sim 12\%$) than for dwarf galaxies in denser regions.   We 
also estimate the nitrogen and oxygen abundances of 2050 void galaxies and 3883 
galaxies in denser regions with $M_r > -20$.  These somewhat brighter galaxies 
(but still fainter than $L_*$) also display similar minor shifts in the N/O 
ratio.  The shifts in the average and median element abundance values in all 
absolute magnitude bins studied are in the same direction, suggesting that the 
large-scale environment may influence the chemical evolution of galaxies.  We 
discuss possible causes of such a large-scale environmental dependence of the 
chemical evolution of galaxies, including retarded star formation and a higher 
ratio of dark matter halo mass to stellar mass in void galaxies.

% Paper 3

% Small-scale enivronment

% GV

\end{abstract}

\iffinal{}{\newpage}

\end{preamble}
