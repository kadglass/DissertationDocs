\chapter[Conclusions]{Conclusions and suggestions for future work}\label{ch:conclusion}

% Papers 1-3
\section[Chemical abundances]{Large-scale environmental influence on the gas-phase chemical abundances in dwarf galaxies}

We find that the large-scale environment influences the chemical evolution of 
dwarf galaxies by estimating the gas-phase oxygen (O/H) and nitrogen (N/H) 
abundances and the N/O ratio of star-forming dwarf galaxies in SDSS DR7 using 
the Direct $T_e$ method and spectroscopic line flux measurements as reprocessed 
in the MPA-JHU catalog \citep{Douglass17a,Douglass17b,Douglass17c}.  Due to the 
minimum redshift limit for detecting the [\ion{O}{2}] $\lambda$3727 emission 
line in SDSS DR7 spectra, only 135 star-forming dwarf galaxies are available for 
analysis with the Direct $T_e$ method.  With this sample, we see no difference 
in the metallicity (O/H) between dwarf galaxies in voids and denser regions 
\citep{Douglass17a}.  Upon closer inspection, we find minor shifts in the 
distributions of gas-phase chemical abundances for void dwarf galaxies when 
compared to dwarf galaxies in denser regions: star-forming void dwarf galaxies 
have higher oxygen abundances (O/H), lower nitrogen abundances (N/H), and lower 
N/O ratios \citep{Douglass17b}.

% Shift in O/H
By deriving a relation between the doubly-ionized oxygen and total oxygen 
abundance in star-forming galaxies, we can expand our sample to consist of 1920 
star-forming dwarf galaxies by removing the dependence on the [\ion{O}{2}] 
$\lambda$3727 doublet.  This larger sample exhibits the same shifts in the 
gas-phase chemical abundances as the smaller, 135 star-forming dwarf galaxy 
sample but with a much higher statistical significance.  We find that 
star-forming void dwarf galaxies have, on average, 7\% higher oxygen abundances 
than star-forming dwarf galaxies in denser regions.  This shift in the gas-phase 
oxygen abundance distribution could be observational evidence for delayed star 
formation in void galaxies when compared to those in denser regions.  This would 
result in a smaller ratio of stellar mass to dark matter halo mass in void 
galaxies than in dwarf galaxies in denser regions, as predicted in simulations 
by \cite{Jung14} and \cite{Tonnesen15}.  If the void galaxies are retaining more 
oxygen as a result of their deeper potential wells (relative to galaxies of the 
same stellar mass in dense regions), then they would be able to commence 
secondary nitrogen synthesis earlier.  We find that some of the star-forming 
dwarf galaxies have a constant value of N/O for a range of masses while the 
remainder exhibit a positive relationship between the stellar mass and N/O ratio 
in the $M_*$-N/O relationship.  We make note of a different critical mass in the 
$M_*$-N/O relationship in the two environments, where the void dwarf galaxies 
begin to show evidence of secondary nitrogen synthesis at a stellar mass 
$\sim$0.4 dex lower than for dwarf galaxies in denser regions.  This indicates 
that the void dwarf galaxies begin to synthesize secondary nitrogen at a smaller 
stellar mass than dwarf galaxies in denser regions.

% Shift in N/O
We also find that star-forming void dwarf galaxies have 17\% lower N/O ratios 
than star-forming dwarf galaxies in denser regions.  This shift towards lower 
N/O ratios in the star-forming dwarf galaxies might be evidence that 
cosmological downsizing is environmentally dependent as predicted by 
\cite{Cen11}.  Our results provide evidence for delayed, ongoing star formation 
and cool gas available to fuel star formation up to the present epoch in void 
dwarf galaxies whose dark matter halos ceased coalescing earlier than for dwarf 
galaxies in denser regions.  We surmise that the differences in the 
distributions of metallicity and the N/O ratio seen in the sample of 
star-forming dwarf galaxies are due to a large-scale ($\sim$10 \hMpc) 
environmental influence on their star formation history and evolution.  

% HI mass
In addition to studying the large-scale environmental influence on the 
distribution of gas-phase chemical abundances in star-forming dwarf galaxies, we 
also investigate if and how the large-scale environment affects relationships 
between the chemical abundances and various other physical properties of the 
galaxies.  Most of our star-forming dwarf galaxies follow the established 
mass-metallicity relation \citep[e.g.,][]{Tremonti04}.  No relationship is 
observed between the metallicity and \ion{H}{1} mass in our sample of 
star-forming dwarf galaxies, but we find that the N/O ratio decreases with 
increasing \ion{H}{1} mass for a given stellar mass.  Because most of our dwarf 
galaxies are in the low metallicity tail of the mass-metallicity relation and 
there is a strong correlation between stellar mass and \ion{H}{1} mass, it is 
not surprising that we see no relationship between the metallicity and 
\ion{H}{1} mass.  The inverse relationship between the N/O ratio and the 
\ion{H}{1} mass is due to the time delay in the production of nitrogen relative 
to oxygen.

% Color
The star-forming dwarf galaxies exhibit an increase in the metallicity and N/O 
ratio with increasing color (both $u-r$ and $g-r$).  Older, redder galaxies have 
had more time to convert their hydrogen into heavier elements through star 
formation, thus increasing their metallicities.  The correlation between the N/O 
ratio and the color of a galaxy indicates a time delay between the release of 
oxygen and nitrogen.

% (s)SFR
We see very little correlation with SFR for either metallicity or the N/O ratio 
in dwarf galaxies, but the metallicity and N/O ratio decrease with increasing 
sSFR.  While we expect a relationship between the SFR and the metallicity due to 
the fundamental relationship, our range of metallicity values might not extend 
high enough to detect a significant relationship with the SFR in our sample of 
star-forming dwarf galaxies.  The strong anti-correlation with low dispersion 
between the N/O ratio and the sSFR in our dwarf galaxies might be evidence that 
oxygen in produced in higher mass stars than those which synthesize nitrogen.

%Beyond the large-scale 
%environmental influence on the distributions of the gas-phase chemical 
%abundances and the critical mass in the $M_*$-N/O relationship, we do not 
%observe any significant differences between the star-forming void and wall dwarf 
%galaxies in any of these relationships.

It has been hypothesized that a special population of extremely metal-poor 
galaxies exist in void regions.  We find no evidence for a special population of 
extremely metal-poor star-forming dwarf galaxies in the voids, as we note an 
equal fraction of low metallicity dwarf galaxies in both the void and denser 
regions.  Due to their low gas-phase oxygen abundances, these 287 dwarf galaxies 
have some of the higher N/O ratios of the star-forming dwarf galaxies sample 
studied.  While the metallicities of these galaxies cause them to stand out in 
the relationships between metallicity and other physical quantities (stellar 
mass, color, (s)SFR), they are not unusual when studying the relationships 
between the N/O ratio and these other quantities.

% Small-scale environment
\section[Small-scale environment]{Small-scale environmental influence on dwarf galaxy evolution}
In addition to the influence of the large-scale environment on the evolution of 
dwarf galaxies, we also study the effects of the small-scale environment on dwarf 
galaxies.  We find that only the presence of a neighboring galaxy within 0.05 
\hMpc or 0.1$r_{vir}$, or the presence of a group within 0.05 \hMpc influences a 
dwarf galaxy's evolution.  Dwarf galaxies with a nearest neighbor galaxy within 
0.05 \hMpc or 0.1$r_{vir}$ tend to be bluer, have a higher sSFR, have higher 
oxygen abundances, and have lower N/O ratios than average.  This matches the 
results of \cite{Park09}, who find that higher luminosity late-late type galaxy 
pairs within 0.05$r_{vir}$ are bluer and have higher SFRs.  In contrast, dwarf 
galaxies within 0.05 \hMpc of the center of the closest group have lower oxygen 
and nitrogen abundances than average.  These results are independent of both the 
maximum relative velocity required to define a nearest neighbor, but they may 
depend on the sample (star-forming versus all dwarf galaxies).

When we incorporate the redshift into the distance calculations, we find that 
the galaxies within 0.1$r_{vir}$ are most likely strongly interacting or merging 
with their nearest neighbor.  These merging galaxies likely share the same dark 
matter halo, suggesting that the dark matter halo is more influential on a 
galaxy's evolution than the distance between a galaxy and its nearest neighbor.

% GV galaxies
\section[GV galaxies]{Properties of green valley galaxies}
Finally, we also investigate the properties of galaxies residing in the green 
valley of the color-magnitude diagram (CMD) in an effort to understand the role 
of the environment on a galaxy's evolution through the color-magnitude diagram.  
We discover that we can define galaxies which are transitioning through the 
green valley of the CMD by combining the galaxy's color, color gradient, and 
inverse concentration index.  Galaxies with a value aimc $< 25$ for the 
morphological type as calculated in the KIAS-VAGC exist in the green valley 
portion of the UV-optical CMD.  Galaxies defined as early types (morphological 
type values equal to 1 or 2) are in the red cloud, while those with 
morphological type values greater than 25 are in the blue sequence.  With this 
quantitative definition for galaxies transitioning through the green valley, we 
can begin to understand the properties of green valley galaxies and their 
evolution.

Based on our analysis, green valley galaxies have stellar masses comparable to 
galaxies in the red cloud.  They have intermediate SFRs and low-to-intermediate 
sSFRs.  Green valley galaxies also have high gas-phase nitrogen abundances 
(N/H), resulting in high N/O ratios.  While their SFRs show that their star 
formation has been quenched, their stellar masses inform us that this is not due 
to any premature quenching mechanism.  The high nitrogen abundances in the green 
valley galaxies indicate that the galaxies are either no longer forming stars 
(since nitrogen is produced in lower mass stars than oxygen), or that the 
galaxies were able to synthesize both primary and secondary nitrogen (heavy 
elements were present during the last few star formation episodes to permit the 
CNO cycle to commence earlier).  This chemical abundance pattern of normal 
oxygen (O/H), high nitrogen (N/H), and high N/O ratio is also seen in galaxies 
classified as AGN, indicating that galaxies in the green valley may have an AGN.

The fraction of void galaxies that are found in the green valley is 
significantly smaller than the fraction of green valley galaxies found in denser 
regions.  Combined with the previous results of the large-scale environmental 
influence on the gas-phase chemical abundances, this indicates that void dwarf 
galaxies may be less evolved than dwarf galaxies in denser environments.


\section[Future work]{Suggestions for future work}

% Metallicity
%  - gradients
%  - chemical abundances of composite, AGN galaxies
%  - SHELS?

% GV galaxies
%  - understand aimc calculation
%  - metallicity gradients
%  - small-scale environment

A critical problem in galaxy formation is understanding how galaxies transition 
from the blue sequence to the red cloud in the optical color-magnitude diagram.  
Star formation is thought to be quenching in galaxies moving through the green 
valley, but the relevant baryonic processes (gas cooling, feedback, etc.) are 
very complex and heavily dependent.  The enrichment of the interstellar medium 
(ISM) and circumgalactic medium (CGM) of a galaxy involves many complicated 
processes, the interplay of which is not yet well understood.  Investigating the 
star formation history and chemical evolution of galaxies in the green valley 
should provide clues of the movement of a galaxy through the color-magnitude 
diagram.

With the advent of 2-dimensional spectroscopic data from SDSS MaNGA \citep{MaNGA,
SDSS13}, it is now possible to examine how the ISM and CGM are enriched over 
time.  MaNGA is the first large-scale survey using integral-field spectroscopy, 
which permits the study of spatially-resolved physical properties of galaxies 
across many different environments, morphologies, and stages of evolution.  In 
particular, it would be instructive to study the gradients of the gas-phase 
chemical abundances of galaxies in order to better understand their evolution 
through the CMD and also the influence of the environment in both large and 
small scales on the chemical evolution of these galaxies.

We find that the N/O ratio decreases with increasing metallicity rather than a 
constant value for the N/O ratio as a function of O/H as observed in other 
studies.  This variation of the behavior of the N/O ratio is contrary to what we 
see in the $M_*$-N/O relationship, where our dwarf galaxies exhibit either a 
constant value (nitrogen plateau) or an increase in (secondary nitrogen 
production) the N/O ratio for increasing stellar mass.  It appears as though the 
extremely low-metallicity galaxies present in the metallicity-N/O relationship 
are causing this unusual behavior.  Detailed observations and high-resolution 
spectra are needed of this population of galaxies, to confirm their metallicity 
estimates and gain insight into this unique relationship.

A better knowledge of the chemical abundances of galaxies with an AGN is also 
crucial to understanding the evolution of galaxies.  Understanding how the 
presence of an AGN affects a galaxy's evolution is currently limited --- at the 
moment, most research is concentrated on the physics within an AGN.  Many 
simulations have been done to test different theories about the role of an AGN 
in a galaxy, and observations are now needed to test the results of these 
simulations.

% Small-scale environment
%  - relative parameter values as a function of distance
%  - visual inspection of mergers

The work done here on the small-scale environment is only the beginning of what 
can be done.  In addition to studying the relationship between a galaxy's 
physical parameters and its distance to the nearest neighbor, it is also 
important to look at the ratio of the galaxy's parameters to its neighbor's and 
the distance to the nearest neighbor --- ``galactic conformity.''  These 
comparisons should include a morphological comparison, similar to the analysis 
done by \cite{Park09}.  We also find evidence of possible mergers when studying 
the influence of the small-scale environment.  Visual inspection of these dwarf 
galaxies is imperative to verify this conclusion.