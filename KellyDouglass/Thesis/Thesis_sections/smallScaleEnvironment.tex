\chapter[Small-scale environment]{The influence of the small-scale environment on dwarf galaxy evolution}


This work was done in collaboration with Daniele Schneider for her senior thesis 
``The effects of small-scale environment on dwarf galaxies.''


%%%%%%%%%%%%%%%%%%%%%%%%%%%%%%%%%%%%%%%%%%%%%%%%%%%%%%%%%%%%%%%%%%%%%%%%%%%%%%%%
%%%%%%%%%%%%%%%%%%%%%%%%%%%%%%%%%%%%%%%%%%%%%%%%%%%%%%%%%%%%%%%%%%%%%%%%%%%%%%%%


\section[Introduction]{Introduction}



%%%%%%%%%%%%%%%%%%%%%%%%%%%%%%%%%%%%%%%%%%%%%%%%%%%%%%%%%%%%%%%%%%%%%%%%%%%%%%%%
%%%%%%%%%%%%%%%%%%%%%%%%%%%%%%%%%%%%%%%%%%%%%%%%%%%%%%%%%%%%%%%%%%%%%%%%%%%%%%%%


\section[Theory]{Calculating distances}\label{sec:Theory_dist}


\subsection{Peculiar velocity}
% Finger of God


\subsection{Sky separation in \hMpc}


\subsection{Fractional virial radii}
% Virial radius calculation


\subsection{Absolute distance to nearest neighbor}


%%%%%%%%%%%%%%%%%%%%%%%%%%%%%%%%%%%%%%%%%%%%%%%%%%%%%%%%%%%%%%%%%%%%%%%%%%%%%%%%
%%%%%%%%%%%%%%%%%%%%%%%%%%%%%%%%%%%%%%%%%%%%%%%%%%%%%%%%%%%%%%%%%%%%%%%%%%%%%%%%


\section[Data]{SDSS Data and galaxy selection}

% SDSS

% photometry from KIAS-VAGC
% sSFR from MPA-JHU (Brinchmann04)
% gas-phase chemical abundances from Douglass17c


\subsection{Various samples}

% Spectroscopic sample (those with chemical abundance estimates)
% Full SDSS dwarf galaxy sample


\subsection{Group catalog}

\cite{Berlind06}


\subsection{Finding the nearest neighbor}
% Absolute magnitude distribution of neighbors


%%%%%%%%%%%%%%%%%%%%%%%%%%%%%%%%%%%%%%%%%%%%%%%%%%%%%%%%%%%%%%%%%%%%%%%%%%%%%%%%
%%%%%%%%%%%%%%%%%%%%%%%%%%%%%%%%%%%%%%%%%%%%%%%%%%%%%%%%%%%%%%%%%%%%%%%%%%%%%%%%


\section[Results]{Distance analysis and results}

Our primary objective is to compare various physical characteristics of dwarf 
galaxies against their nearest neighbors or groups to discern how the 
small-scale environment affects their evolution.


\subsection{Parameter -- distance relations}\label{sec:Relations}

The relationships between various physical parameters and the distance to the 
nearest neighbor or group for our sample of star-forming dwarf galaxies are 
shown in Figs. \ref{fig:ur}--\ref{fig:NO}.  In each of the four plots within 
each figure, we bin the galaxies in distance-space and show the average 
parameter value in each bin, to see any overall trends in the data.  We also fit 
a linear line to the data; the output of these fits are in Table \ref{tab:fits}.  
Each of these four plots probes this distance relationship from a different 
angle.  The top left plots in Figs. \ref{fig:ur}--\ref{fig:NO} compare the 
target dwarf galaxy's parameter to the distance to its nearest neighbor, as 
defined by the the closest galaxy on the sky in \hMpc with a velocity difference 
less than 300 km/s.  The bottom left plots compare the target dwarf galaxy's 
parameter to the distance to its nearest neighbor, as defined by the closest 
galaxy on the sky in units of the virial radius of the neighbor galaxy with a 
velocity difference less than 300 km/s.  The nearest galaxy is not necessarily 
the same for these two distance measurements, as explained in Sec. 
\ref{sec:Theory_dist}.

We repeat this same analysis on the nearest groups to the target dwarf galaxy.  
The top right plots in Figs. \ref{fig:ur}--\ref{fig:NO} compare the target dwarf 
galaxy's parameter to the distance to the center of the nearest group, as 
defined as the closest group on the sky in \hMpc with a velocity difference less 
than 300 km/s.  The bottom right plots in Figs. \ref{fig:ur}--\ref{fig:NO} 
compare the target dwarf galaxy's parameter to the distance to the nearest group 
as defined by the closest group on the sky in units of the group's rms radius 
with a velocity difference of less than 300 km/s.  Groups are very rare in the 
void environment (by nature of the void environment), so the uncertainty in the 
binned values shown in these plots is much larger than in the galaxy neighbor 
plots.

These figures reveal influences from both the large-scale and small-scale 
environments, since we are identifying which galaxies reside in void regions and 
which do not.


\subsubsection{Color}

\begin{figure}
    \includegraphics[width=0.49\textwidth]{Images/smallScaleEnvironment/dwarf_ur_300}
    \includegraphics[width=0.49\textwidth]{Images/smallScaleEnvironment/dwarf_ur_300_group}
    \caption[$u-r$ versus distance to nearest neighbor and group]{Color ($u-r$) 
    versus distance to the nearest galaxy (on the left) and nearest group (on 
    the right).  The top panel shows the color as a function of the sky 
    separation in \hMpc between the target dwarf galaxy and the neighbor, while 
    the bottom panel shows the color as a function of the closest virial 
    neighbor.  Void galaxies are shown in red, while wall galaxies are shown in 
    black and unknown in green.  We have also included the average color for the 
    galaxies after binning by distance, to discern any finer behavior in the 
    relationships.  Linear fits to the void and wall galaxies are shown in 
    orange and blue, respectively.  It is clear that the void dwarf galaxies are 
    bluer than the wall dwarf galaxies.  The nearest galaxies appear to only 
    have some affect on the dwarf galaxy's color at separations less than 0.05 
    \hMpc, or 0.05$r_{vir}$.  The closest group appears to have some 
    affect on the target dwarf galaxy's color at distances less than 0.1 \hMpc 
    to the group's center; there appears to be no relationship between a dwarf 
    galaxy's color and its distance to the nearest group as represented by the 
    fraction of the group's radius.}
    \label{fig:ur}
\end{figure}

Because of the known morphology-density relation, we excepted to find that a 
dwarf galaxy's color became more blue as the distance to its nearest neighbor 
increased.  Fig. \ref{fig:ur} shows very little relationship between the 
distance and color, except at the smallest distance bin.  The linear fits 
quantify this observation --- the slopes are on the order of 10$^{-3}$.  
However, within a distance of 0.05 \hMpc or 0.05$r_{vir}$, dwarf galaxies tend 
to be bluer than at further distances from their nearest neighbor.  At distances 
less than 0.1 \hMpc from the center of the nearest groups, the dwarf galaxies 
are redder than average.  However, there does not appear to be any relationship 
between a dwarf galaxy's color and its distance to the center of the nearest 
group in units of the group's rms radius.  It has been well-established that 
void galaxies tend to be bluer than galaxies in denser environments 
\citep{Grogin99,Rojas04,Patiri06,vonBendaBeckmann08,Hoyle12}; this shift is 
apparent in Fig. \ref{fig:ur}, where the void dwarf galaxies are slightly bluer 
than the wall dwarf galaxies.  


\subsubsection{sSFR}

\begin{figure}
    \includegraphics[width=0.49\textwidth]{Images/smallScaleEnvironment/dwarf_sSFR_300}
    \includegraphics[width=0.49\textwidth]{Images/smallScaleEnvironment/dwarf_sSFR_300_group}
    \caption[sSFR versus distance to nearest neighbor and group]{sSFR versus 
    distance to the nearest galaxy (on the left) and nearest group (on the 
    right).  The top panel shows the sSFR as a function of the sky separation in 
    \hMpc between the target dwarf galaxy and the neighbor, while the bottom 
    panel shows the sSFR as a function of the closest virial neighbor.  Void 
    galaxies are shown in red, while wall galaxies are shown in black and 
    unknown in green.  We have also included the average sSFR for the galaxies 
    after binning by distance, to discern any finer behavior in the 
    relationships.  Linear fits to the void and wall galaxies are shown in 
    orange and blue, respectively.  It is clear that the void dwarf galaxies 
    have higher sSFRs than the wall dwarf galaxies.  Only the neighbor galaxies 
    at separations less than 0.05 \hMpc or 0.05$r_{vir}$ appear to have some 
    affect on the dwarf galaxies' sSFR.  There appears to be no relationship 
    between a dwarf galaxy's sSFR and its distance to the nearest group.}
    \label{fig:sSFR}
\end{figure}

Following our prediction for the color-distance relations, we expect that the 
sSFR to increase with distance from the nearest neighbor.  We also see very 
little relationship between the distance and sSFR in Fig. \ref{fig:sSFR}, except 
in the smallest distance bin.  The linear fits quantify this observation --- the 
slopes are on the order of 10$^{-3}$.  Within a distance of 0.05 \hMpc or 
0.05$r_{vir}$, dwarf galaxies tend to have higher sSFRs than at further 
distances from their nearest neighbor.  There does not appear to be any 
relationship between a dwarf galaxy's sSFR and its distance to the center of the 
nearest group in either distance metric.  In these distance comparisons, there 
is a shift towards higher sSFRs in the void dwarf galaxies when compared to the 
wall dwarf galaxies, as has been observed many times before \citep{Rojas05,
vonBendaBeckmann08,Moorman15,Beygu16}.  


\subsubsection{Metallicity}

\begin{figure}
    \includegraphics[width=0.49\textwidth]{Images/smallScaleEnvironment/dwarf_OH_300}
    \includegraphics[width=0.49\textwidth]{Images/smallScaleEnvironment/dwarf_OH_300_group}
    \caption[Metallicity versus distance to nearest neighbor and group]
    {Metallicity versus distance to the nearest galaxy (on the left) and nearest 
    group (on the right).  The top panel shows the metallicity as a function of 
    the sky separation in \hMpc between the target dwarf galaxy and the 
    neighbor, while the bottom panel shows the metallicity as a function of the 
    closest virial neighbor.  Void galaxies are shown in red, while wall 
    galaxies are shown in black and unknown in green.  We have also included the 
    average metallicity for the galaxies after binning by distance, to discern 
    any finer behavior in the relationships.  Linear fits to the void and wall 
    galaxies are shown in orange and blue, respectively.  It is clear that the 
    void dwarf galaxies have higher metallicities than the wall dwarf galaxies.  
    Only the neighbor galaxies at separations less than 0.05 \hMpc or 
    0.05$r_{vir}$ appear to have some affect on the dwarf galaxies' metallicity.  
    When measuring the distance to the nearest group in \hMpc, it appears that 
    the dwarf galaxies have lower metallicities at distances less than 0.1 \hMpc 
    from their nearest group.}
    \label{fig:OH}
\end{figure}

Based on our hypothesis that galaxies would be redder and have lower sSFRs at 
small distances to their nearest neighbors, we anticipated that the metallicity 
of the galaxies would decrease with increasing distance.  As before, we only see 
a relationship between the distance and metallicity in the smallest distance 
bin in Fig. \ref{fig:OH}.  The linear fits quantify this observation --- the 
slopes are on the order of 10$^{-2}$.  Within a distance of 0.05 \hMpc or 
0.05$r_{vir}$, dwarf galaxies tend to have higher metallicities than at further 
distances from their nearest neighbor.  At distances less than 0.1 \hMpc from 
the center of the nearest group, dwarf galaxies might have lower than average 
metallicities.  However, this could be an erroneous conclusion due to low-number 
statistics.  In these distance comparisons, there is a shift towards higher 
metallicities in the void dwarf galaxies when compared to the wall dwarf 
galaxies, as has been observed by \cite{Douglass17b} and Douglass \& Vogeley 
(2017, in prep).  


\subsubsection{Nitrogen abundance}

\begin{figure}
    \includegraphics[width=0.49\textwidth]{Images/smallScaleEnvironment/dwarf_NH_300}
    \includegraphics[width=0.49\textwidth]{Images/smallScaleEnvironment/dwarf_NH_300_group}
    \caption[N/H versus distance to nearest neighbor and group]{Gas-phase 
    nitrogen abundance versus distance to the nearest galaxy (on the left) and 
    nearest group (on the right).  The top panel shows N/H as a function of the 
    sky separation in \hMpc between the target dwarf galaxy and the neighbor, 
    while the bottom panel shows N/H as a function of the closest virial 
    neighbor.  Void galaxies are shown in red, while wall galaxies are shown in 
    black and unknown in green.  We have also included the average nitrogen 
    abundance for the galaxies after binning by distance, to discern any finer 
    behavior in the relationships.  Linear fits to the void and wall galaxies 
    are shown in orange and blue, respectively.  There is very little difference 
    in N/H between the two large-scale environments.  Only the neighbor galaxies 
    at separations less than 0.05$r_{vir}$ appear to have some affect on the 
    dwarf galaxies' nitrogen abundance.  There does not appear to be any 
    relationship between the dwarf galaxies' nitrogen abundance and the distance 
    to their nearest group.}
    \label{fig:NH}
\end{figure}

It was expected that the gas-phase nitrogen abundance would follow the same 
trend as the metallicity, to decrease with increasing distance.  As observed 
with the other parameters, Fig. \ref{fig:NH} shows very little relationship 
between the distance and N/H, except in the smallest distance bin for the 
closest virial neighbor.  The linear fits quantify this observation --- the 
slopes are on the order of 10$^{-2}$.  Within a distance of 0.05$r_{vir}$, dwarf 
galaxies tend to have higher nitrogen abundances than at distances further from 
their nearest neighbor.  There does not seem to be any relationship between the 
nitrogen abundance of the star-forming dwarf galaxies and distance to the center 
of the nearest group.  Unlike the shifts seen with the other parameters and what 
is observed in \cite{Douglass17b} and Douglass \& Vogeley (2017, in prep), we 
see no significant difference in the nitrogen abundance resulting from the 
large-scale environment.


\subsubsection{N/O ratio}

\begin{figure}
    \includegraphics[width=0.49\textwidth]{Images/smallScaleEnvironment/dwarf_NO_300}
    \includegraphics[width=0.49\textwidth]{Images/smallScaleEnvironment/dwarf_NO_300_group}
    \caption[N/O versus distance to nearest neighbor and group]{N/O ratio versus 
    distance to the nearest galaxy (on the left) and nearest group (on the 
    right).  The top panel shows the N/O ratio as a function of the sky 
    separation in \hMpc between the target dwarf galaxy and the neighbor, while 
    the bottom panel shows N/O as a function of the closest virial neighbor.  
    Void galaxies are shown in red, while wall galaxies are shown in black and 
    unknown in green.  We have also included the average nitrogen abundance for 
    the galaxies after binning by distance, to discern any finer behavior in the 
    relationships.  Linear fits to the void and wall galaxies are shown in 
    orange and blue, respectively.  The void dwarf galaxies have lower N/O 
    ratios than the wall dwarf galaxies, but there is no distinct relationship 
    between the distance to the nearest neighbor and the N/O ratio.  The N/O 
    ratio might be higher in dwarf galaxies within 0.05 \hMpc of the center of 
    the closest group.}
    \label{fig:NO}
\end{figure}

We do not expect any influence on the relative synthesis of oxygen and nitrogen 
from the proximity to a nearest neighbor.  Unlike the other parameters studied, 
Fig. \ref{fig:NO} shows that the N/O ratio does not have any relationship with 
the distance to a nearest neighbor at any separation.  This is reflected in the 
linear fits to the data --- the slopes are on the order of 10$^{-2}$ and 
smaller.  When looking at the relationship between the N/O ratio and the 
distance to the nearest group, though, the N/O ratio might be higher than 
average in galaxies within 0.1 \hMpc of the group's center.  The shift towards 
lower N/O ratios in star-forming void dwarf galaxies is readily apparent, as 
\cite{Douglass17b} and Douglass \& Vogeley (2017, in prep) find.


\subsection{Linear fit parameters}

\begin{table}
    \begin{tabular}{lcccc}
        Property & Slope (wall) & Slope (void) & Intercept (wall) & Intercept (void)\\
        \hline
        \hline
        \multicolumn{5}{c}{Nearest galaxy by distance}\\
        \hline
        $u-r$ & $1.25\pm 1.17\times 10^{-3}$ & $-9.14\pm 0.56\times 10^{-3}$ & $1.15\pm 0.00076$ & $1.09\pm0.00049$\\
        sSFR & $2.07\pm 0.18\times 10^{-2}$ & $-0.16\pm 0.08\times 10^{-3}$ & $-9.45\pm 0.001$ & $-9.34\pm 0.0007$\\
        \OH & $5.35\pm 0.17\times 10^{-2}$ & $0.33\pm 0.08\times 10^{-2}$ & $7.75\pm 0.0011$ & $7.86\pm 0.0007$\\
        \NH & $1.64\pm 0.11\times 10^{-2}$ & $0.62\pm 0.06\times 10^{-2}$ & $6.53\pm 0.0007$ & $6.52\pm 0.0006$\\
        \NO & $-3.70\pm 0.13\times 10^{-2}$ & $0.29\pm 0.07\times 10^{-2}$ & $-1.23\pm 0.0009$ & $-1.35\pm 0.0006$\\
        \hline
        \multicolumn{5}{c}{Nearest galaxy by fraction of virial radius}\\
        \hline
        $u-r$ & $-3.81\pm 0.04\times 10^{-3}$ & $-0.46\pm 0.01\times 10^{-3}$ & $1.16\pm 0.00014$ & $1.09\pm0.00008$\\
        sSFR & $5.96\pm 0.06\times 10^{-3}$ & $-0.41\pm 0.02\times 10^{-3}$ & $-9.45\pm 0.0002$ & $-9.33\pm 0.0001$\\
        \OH & $1.41\pm 0.005\times 10^{-2}$ & $0.12\pm 0.08\times 10^{-2}$ & $7.74\pm 0.0002$ & $7.86\pm 0.0001$\\
        \NH & $1.82\pm 0.11\times 10^{-3}$ & $-1.14\pm 0.002\times 10^{-3}$ & $6.53\pm 0.0001$ & $6.52\pm 0.0001$\\
        \NO & $-1.22\pm 0.004\times 10^{-2}$ & $-0.41\pm 0.02\times 10^{-3}$ & $-1.21\pm 0.0002$ & $-1.33\pm 0.0001$\\
        \hline
        \multicolumn{5}{c}{Nearest group by distance}\\
        \hline
        $u-r$ & $-2.53\pm 0.01\times 10^{-3}$ & $1.34\pm 0.01\times 10^{-3}$ & $1.16\pm 0.00008$ & $1.08\pm0.0001$\\
        sSFR & $7.77\pm 0.02\times 10^{-3}$ & $-1.68\pm 0.02\times 10^{-3}$ & $-9.48\pm 0.001$ & $-9.33\pm 0.0001$\\
        \OH & $3.65\pm 0.16\times 10^{-4}$ & $2.06\pm 0.19\times 10^{-4}$ & $7.77\pm 0.0001$ & $7.86\pm 0.0001$\\
        \NH & $4.84\pm 0.01\times 10^{-3}$ & $6.30\pm 0.01\times 10^{-3}$ & $6.51\pm 0.00008$ & $6.49\pm 0.0001$\\
        \NO & $4.47\pm 0.01\times 10^{-3}$ & $6.09\pm 0.01\times 10^{-3}$ & $-1.26\pm 0.0001$ & $-1.38\pm 0.0001$\\
        \hline
        \multicolumn{5}{c}{Nearest group by fraction of group radius}\\
        \hline
        $u-r$ & $-1.00\pm 0.002\times 10^{-3}$ & $0.90\pm 0.002\times 10^{-3}$ & $1.16\pm 0.00003$ & $1.07\pm0.00003$\\
        sSFR & $2.70\pm 0.002\times 10^{-3}$ & $-1.71\pm 0.002\times 10^{-3}$ & $-9.48\pm 0.00005$ & $-9.31\pm 0.00005$\\
        \OH & $8.81\pm 0.02\times 10^{-4}$ & $-8.74\pm 0.19\times 10^{-4}$ & $7.76\pm 0.00005$ & $7.88\pm 0.00005$\\
        \NH & $2.31\pm 0.001\times 10^{-3}$ & $2.47\pm 0.001\times 10^{-3}$ & $6.49\pm 0.00003$ & $6.47\pm 0.00004$\\
        \NO & $1.43\pm 0.002\times 10^{-3}$ & $3.35\pm 0.002\times 10^{-3}$ & $-1.26\pm 0.00005$ & $-1.40\pm 0.00005$
    \end{tabular}
    \caption[Fit parameters of properties versus distances]{Linear fit 
    parameters to various properties of the target dwarf galaxies by their 
    distances to the nearest galaxy in units of \hMpc, the nearest galaxy in 
    units of the neighbor's virial radius, the center of the nearest group in 
    units of \hMpc, and the nearest group in units of the group's rms radius; 
    all objects must be within 300 km/s of the target galaxy.  The slopes are 
    all negligible, quantifying the observations made that the proximity to a 
    galaxy or group has little influence on a dwarf galaxy's evolution.}
    \label{tab:fits}
\end{table}

To quantify the results shown in Figs. \ref{fig:ur}--\ref{fig:NO}, we calculate 
the parameters for the best fit linear line.  Any slope of significant magnitude 
shows an overall correlation between a given physical parameter and the galaxy's 
distance to its nearest neighbor or group.  The results of this analysis are 
listed in Table \ref{tab:fits}.  These slopes reflect the observations described 
in Section \ref{sec:Relations}: there is no correlation between the distance to 
the nearest neighbor and a galaxy's color, sSFR, or gas-phase chemical 
abundances.  This analysis does not capture any variations within the range of 
distances for all dwarf galaxies included in this study.


\subsection{Selection effects}

% Peculiar velocity
\begin{figure}
    \includegraphics[width=0.49\textwidth]{Images/smallScaleEnvironment/dwarf_ur_150}
    \includegraphics[width=0.49\textwidth]{Images/smallScaleEnvironment/dwarf_ur_600}
    \caption[Sensitivity to peculiar velocity maximum]{Color versus distance to 
    nearest galaxy (in units of \hMpc on top and virial radii on bottom), with a 
    maximum allowed peculiar velocity of 150 km/s in the left panel and 600 km/s 
    in the right panel.  Void galaxies are shown in red, wall in black, and 
    unknown in green.  The galaxies are binned by distance to tease out any 
    trends at smaller distance scales; linear fits to the void and wall galaxies 
    are shown in orange and blue, respectively.  When compared with the two 
    plots in the left panel of Fig. \ref{fig:ur}, we see that there is no 
    significant influence on our results from the choice of maximum peculiar 
    velocity allowed.}
    \label{fig:ur_vpeculiar}
\end{figure}

We test two of our input parameters to understand how sensitive our results are 
to any initial conditions.  The first parameter we discuss is the sensitivity of 
our results on the maximum peculiar velocity to define a match.  Throughout our 
analysis, we use 300 km/s as the maximum velocity separation allowed between the 
target galaxy and its nearest neighbor or group.  We look at how this affects our 
results by repeating the analysis with maximum velocities of 150 km/s and 600 
km/s.  The results of this comparison on the color of the galaxies can be seen in 
Fig. \ref{fig:ur_vpeculiar}.  The nearest neighbors in the left-hand panel are 
restricted to a maximum peculiar velocity of 150 km/s, while those in the right 
panel are restricted to 600 km/s.  When compared to each other and to the 
left-hand panel of Fig. \ref{fig:ur}, it is clear that our choice of maximum 
peculiar velocity has no affect on the results of the study.

% Spectroscopic sample v. complete sample
\begin{figure}
    \includegraphics[width=0.5\textwidth]{Images/smallScaleEnvironment/ALLdwarf_ur_300}
    \caption[Color versus distance of full SDSS dwarf population]{Color versus 
    distance to the nearest galaxy (in units of \hMpc on top and virial radii on 
    the bottom) for the entire dwarf galaxy population in SDSS DR7.  When 
    compared to the left-hand panel of Fig. \ref{fig:ur}, we see that there is 
    no difference in the results of the analysis by studying only star-forming 
    galaxies with sufficient detection of the various emission lines necessary 
    to estimate the gas-phase chemical abundances.}
    \label{fig:ur_allDwarf}
\end{figure}

We also test the sensitivity of our results to the population of galaxies being 
studied.  Because we want to look at the relationship between distance and the 
gas-phase chemical abundances of the dwarf galaxies, our sample is limited to 
star-forming dwarf galaxies with detected emission lines necessary for 
estimation of the chemical abundances with the Direct $T_e$ method 
\citep[see][for more details]{Douglass17a}.  We perform the same distance 
analysis on all dwarf galaxies detected in SDSS DR7 with respect to their 
color, to understand how our results depend on our sample.  When we compare Fig. 
\ref{fig:ur_allDwarf} with the left-hand panel of Fig. \ref{fig:ur}, we see that 
there is no difference in the correlation between color and distance to the 
nearest neighbor.  It is clear that our selection bias to star-forming dwarf 
galaxies does not influence any trends we observe in our analysis.


\subsection{Including redshift in the distance}
% Evidence of mergers

\begin{figure}
    \includegraphics[width=0.5\textwidth]{Images/smallScaleEnvironment/dwarf_ur_xyz}
    \caption[Color versus distance calculated with redshift]{Color versus 
    distance to the nearest galaxy (in units of \hMpc on top and virial radii on 
    the bottom) for the star-forming dwarf galaxies.  The redshift is included 
    when calculating the distance to the nearest neighbor.  While there is still 
    no correlation between distance and color, we do note that there is a gap in 
    the distribution of galaxies around a distance of 0.05 \hMpc or 0.5$r_{vir}$ 
    from the nearest neighbor.}
    \label{fig:ur_xyz}
\end{figure}

Realizing that the peculiar velocity, which is included in a galaxy's redshift, 
is not insignificant at the smaller scales, we have been careful to avoid 
calculating the distance between objects with the redshift.  However, we are 
interested to see how the inclusion of redshift in the distance calculations 
affects the results of our analysis.  Therefore, we repeat the same analysis on 
the relationship between color and distance, but this time we include redshift 
as a third component in the distance calculations.  Consequently, we no longer 
limit our sample by a peculiar velocity separation.  Fig. \ref{fig:ur_xyz} shows 
the relationship between these distances and the color of the star-forming dwarf 
galaxies.  When compared to Fig. \ref{fig:ur}, we see that there is no change in 
the correlation between distance and color for the galaxies.  

We note the existence of a gap in the distribution of galaxies around a distance 
of 0.05 \hMpc or 0.5$r_{vir}$ from the nearest neighbor that is not present in 
Fig. \ref{fig:ur}.  When we incorporate the redshift into the distance 
calculations, the resulting distance between galaxies which have small sky 
separations but larger redshift separations is much larger than those with 
larger sky separations and little redshift separation.  As a result, galaxies 
which were originally close to the color axis in Fig. \ref{fig:ur} will be 
moved to much larger distances, while the locations of those which were 
originally further from the color axis will change much less.  The dwarf 
galaxies which remain close to the color axis in Fig. \ref{fig:ur_xyz} must, 
therefore, have small sky separation and almost no difference in their peculiar 
velocities.  We conclude that these represent merging systems, which visual 
inspection will help to confirm.


%%%%%%%%%%%%%%%%%%%%%%%%%%%%%%%%%%%%%%%%%%%%%%%%%%%%%%%%%%%%%%%%%%%%%%%%%%%%%%%%
%%%%%%%%%%%%%%%%%%%%%%%%%%%%%%%%%%%%%%%%%%%%%%%%%%%%%%%%%%%%%%%%%%%%%%%%%%%%%%%%


\section[Discussion]{Discussion}

We see almost no relationship between a dwarf galaxy's color, sSFR, or gas-phase 
chemical abundance and its distance to the nearest galaxy or group, implying 
that the small-scale environment does not significantly influence a dwarf 
galaxy's evolution.  This is in contrast to the large-scale environment, which 
we have seen to influence the formation and evolution of dwarf galaxies.

Only those galaxies within 0.05 \hMpc and 0.05$r_{vir}$ of a nearest neighbor, 
or those within 0.1 \hMpc of the nearest group appear to deviate from the 
average galaxy values.  The galaxies within this proximity of their nearest 
neighbor tend to be bluer, have a higher sSFR, and have higher oxygen and 
nitrogen abundances.  Based on the shift in the distribution of galaxies in Fig. 
\ref{fig:ur_xyz}, these galaxies might be merging or strongly interacting with 
their nearest neighbor.  This would then provide evidence that galaxy 
interactions result in a burst of star formation, which would increase the 
gas-phase chemical abundances of the dwarf galaxies.  Because merging galaxies 
share the same dark matter halo, it appears that the sharing of a dark matter 
halo has more influence on the evolution of a galaxy than the distance to its 
nearest neighbor.

Dwarf galaxies within 0.1 \hMpc of the center of the nearest group are redder, 
have lower oxygen abundances, and have higher N/O ratios than average.  Being so 
close to the center of a group prevents more recent episodes of star formation.  
Due to their proximity to the group center, it is also likely that these dwarf 
galaxies are not able to retain as much of their heavier elements as for a more 
isolated galaxy, reducing their gas-phase oxygen abundance (and therefore 
increasing their N/O ratio).


\subsection{Comparison to literature results}

The influence on the gas-phase oxygen abundance within 0.05 \hMpc and 
0.05$r_{vir}$ agrees with the results of \cite{Shields91,Pustilnik06,Cooper08,
Ellison09,Pustilnik11a,Pustilnik14}, and \cite{SanchezAlmeida16}, which all find 
that galaxies with higher metallicities preferentially reside in denser regions.  
Work by \cite{Rupke08} finds that interacting galaxies have suppressed 
metallicities due to interaction- or merger-induced gas flows into the galaxy 
centers.

A study combining the effects of interactions and the large-scale environment is 
explained in \cite{Park09}.  They find two characteristic distances within which 
the behavior of the target galaxy changes: 0.05$r_{vir}$ and $r_{vir}$ of the 
neighboring galaxy.  Our results seem to confirm the significance of distances 
out to 0.05$r_{vir}$, while we see no significant change around the virial 
radius of the neighboring galaxy.  While they only look at galaxies with 
$M_r < -19$ and limit the neighbors to be at least half a magnitude brighter 
than the target, \cite{Park09} find that the morphology and luminosity play a 
significant role in these relationships.  Of particular note is their 
observation that star formation increases in late type galaxies when their 
nearest neighbor is also of late type.  This is the same behavior we see in our 
sample of dwarf galaxies at distances less than 0.05$r_{vir}$.  With all our 
target galaxies actively forming stars, and more than half of their nearest 
neighbors also dwarf galaxies, it is most likely that our galaxy pairs are also 
of the late-late type.  Based on the results of \cite{Park09}, the deviations we 
see for those galaxies with nearest neighbors within 0.05 \hMpc and 
0.05$r_{vir}$ warrant further study.


%%%%%%%%%%%%%%%%%%%%%%%%%%%%%%%%%%%%%%%%%%%%%%%%%%%%%%%%%%%%%%%%%%%%%%%%%%%%%%%%
%%%%%%%%%%%%%%%%%%%%%%%%%%%%%%%%%%%%%%%%%%%%%%%%%%%%%%%%%%%%%%%%%%%%%%%%%%%%%%%%


\section[Conclusion]{Conclusion}

Using the star-forming dwarf galaxies in the SDSS DR7 sample with gas-phase 
chemical abundances from Douglass \& Vogeley (2017, in prep), we investigate the 
influence of the small-scale environment on the evolution of dwarf galaxies.  
From the $\sim$2000 galaxies in the sample, there only appears to be an effect 
from a neighboring galaxy within 0.05 \hMpc or 0.05$r_{vir}$.  The proximity of 
a group seems to only affect the target dwarf galaxy if it is within 0.1 \hMpc.  
Thus, the small-scale ($\sim 1$ \hMpc) environment does not appear to strongly 
influence the evolution of dwarf galaxies.

We examine the relationship between distance to the nearest neighbor or group 
and the target galaxy's color, sSFR, and gas-phase chemical abundances.  We find 
that, for those galaxies with a neighbor within 0.05 \hMpc or 0.05$r_{vir}$, the 
dwarf galaxies are bluer, have a higher sSFR, and have higher oxygen and 
nitrogen abundances than average.  In contrast, dwarf galaxies within 0.1 \hMpc 
of the center of the closest group are redder, have lower oxygen abundances, and 
have higher N/O ratios than average.  These results do not depend on the maximum 
peculiar velocity difference or on the sample (star-forming versus all 
galaxies).

When we incorporate the redshift into the distance calculations, we find that 
those galaxies within 0.05$r_{vir}$ are most likely mergers or strongly 
interacting with their nearest neighbor.  This matches the results of 
\cite{Park09}, who find that late-late type galaxy pairs within 0.05$r_{vir}$ 
are bluer and have higher star formation rates.  These merging galaxies likely 
share the same dark matter halo, indicating that the dark matter halo is more 
influential on a galaxy's evolution than its distance to the nearest neighbor.  