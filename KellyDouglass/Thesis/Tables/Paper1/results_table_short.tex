\begin{table}
\centering

\begin{tabular}{cccccccc}
Index\footnote{KIAS-VAGC galaxy index number} & R.A. & Decl. & Redshift & $M_r$ & \multicolumn{2}{c}{$12 + \log \left( \frac{\text{O}}{\text{H}} \right)$} & Void/Wall \\
\hline \\
63713 & \RA{09}{20}{04}{.27} & -\dec{00}{30}{08}{.97} & 0.0257 & -16.73 & 7.80 & $\pm$0.41 & Wall \\
73537 & \RA{09}{25}{24}{.23} & +\dec{00}{12}{40}{.39} & 0.0250 & -16.94 & 7.94 & $\pm$0.34 & Wall \\
75442 & \RA{13}{13}{24}{.25} & +\dec{00}{15}{02}{.95} & 0.0264 & -16.81 & 7.55 & $\pm$0.35 & Void \\
168874 & \RA{11}{45}{13}{.16} & -\dec{01}{48}{17}{.68} & 0.0273 & -16.99 & 8.16 & $\pm$0.31 & Wall \\
184308 & \RA{09}{39}{09}{.38} & +\dec{00}{59}{04}{.15} & 0.0244 & -16.73 & 7.36 & $\pm$0.43 & Wall\\
\end{tabular}

\caption[Dwarf galaxy properties]{Five of the 135 dwarf galaxies analyzed from SDSS DR7.  The flux values for all required emission lines can be found in the MPA-JHU value-added catalog.  Metallicity values are calculated using the direct $T_e$ method, with error estimates via a Monte Carlo method.  The void catalog of \cite{Pan12} is used to classify the galaxies as either Void or Wall.  If a galaxy is located too close to the boundary of the SDSS survey to identify whether or not it is inside a void, it is labeled as Uncertain.  Table \ref{tab:Results_P1} is published in its entirety online in a machine-readable format.  A portion is shown here for guidance regarding its form and content.}

\label{tab:Results_P1}

\end{table}
